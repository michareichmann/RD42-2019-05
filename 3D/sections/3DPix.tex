\section{3D Pixel Detector}
%%%%%%%%%%%%%%%%%%%%%%%%%%%%%%%% FRAME 0 %%%%%%%%%%%%%%%%%%%%%%%%%%%%%%%%%%%%%%%%%%%%
\subsection{3D Detectors}
\begin{frame}{Working Principle}

	\fig{.48}{3DConcept}
	
	\begin{itemize}\itemfill
		\item after large radiation fluence all detectors become trap limited
		\item bias and readout electrode inside detector material
		\item same thickness $D$ \ra same amount of induced charge \ra shorter drift distance $L$
		\item \good{increase collected charge in detectors with limited mean drift path (Schubweg)}
	\end{itemize}

\end{frame}

%%%%%%%%%%%%%%%%%%%%%%%%%%%%%%%% FRAME 1 %%%%%%%%%%%%%%%%%%%%%%%%%%%%%%%%%%%%%%%%%%%%
\begin{frame}{Bump Bonding}

	\subfigs{\subfig[.48][.1]{.3}{BondingScheme}[Bump bond schematics]}{\subfig[.44]{.4}{BBCMS1}[\SI{3x2}{} bump pads]}
	
	\begin{itemize}\itemfill
		\item electrodes (columns) drilled with femto-second laser
		\item connection to bias and readout with surface metallisation
		\item ganging of cells to match pixel pitch of readout-chip (ROC)
		\item small gap (\SI{\sim15}{\micro\meter}) to the surface to avoid a high voltage break-through 
	\end{itemize}

\end{frame}
