\section{Results}
%%%%%%%%%%%%%%%%%%%%%%%%%%%%%%%%%%%%%% FRAME 0 %%%%%%%%%%%%%%%%%%%%%%%%%%%%%%%%%%%%%%%%%%%%%%%%
\begin{frame}{PSI - August 2017}

	\subfigs{\subfig{.5}[r]{PH1}}{\only<1>{\subfig{.5}[r]{EM1}}\only<2>{\subfig{.5}[r]{HE1}}}
	
	\begin{itemize}\itemfill
		\item<1-> pulse height looks OK, but ``pedestal'' of unknown origin (cannot be real)
		\begin{itemize}
			\item<1-> cannot be remeasured, since the ROC was exchanged 
		\end{itemize}
		\item<1-> Langau MPV: \SI{13500}{e}
		\item<1-> uniform efficiency
		\item<2> high efficiency of \SI{99.2\pm.1}{\%}
	\end{itemize}
	
\end{frame}
%%%%%%%%%%%%%%%%%%%%%%%%%%%%%%%%%%%%%% FRAME 1 %%%%%%%%%%%%%%%%%%%%%%%%%%%%%%%%%%%%%%%%%%%%%%%%
\begin{frame}{PSI - Oct 2018}

	\subfigs{\subfig{.5}[r]{PH2}}{\only<1>{\subfig{.5}[r]{EM2}}\only<2>{\subfig{.5}[r]{HE2}}}
	
	\begin{itemize}\itemfill
		\item<1-> left part of pulse height distribution not understood
		\item<1-> Langau MPV: \SI{8000}{e}
		\item<1-> efficiency much less uniform \ra already loose bumps?
		\item<2> lower efficiency of \SI{97.3\pm.2}{\%}
	\end{itemize}
	
\end{frame}
%%%%%%%%%%%%%%%%%%%%%%%%%%%%%%%%%%%%%% FRAME 2 %%%%%%%%%%%%%%%%%%%%%%%%%%%%%%%%%%%%%%%%%%%%%%%%
\begin{frame}{CERN - Oct 2018}

	\subfigs{\subfig{.5}[r]{Occ3}}{\subfig{.5}[r]{Occ3-1}}
	
	\begin{itemize}\itemfill
		\item tried different calibrations of the chip
		\item using the same region as at PSI 
		\item also small region with 3D cells without bump-bonding (rows 76-79)
	\end{itemize}
	
\end{frame}
%%%%%%%%%%%%%%%%%%%%%%%%%%%%%%%%%%%%%% FRAME 3 %%%%%%%%%%%%%%%%%%%%%%%%%%%%%%%%%%%%%%%%%%%%%%%%
\begin{frame}{CERN - Oct 2018 (0)}

	\subfigs{\subfig{.5}[r]{PH4}}{\subfig{.5}[r]{F4}}
	
	\begin{itemize}\itemfill
		\item calibration on the bench for single plane, operation with three planes
		\item error fit to demonstrate the calibration for a single pixel
		\item calibration clearly wrong
	\end{itemize}
	
\end{frame}
%%%%%%%%%%%%%%%%%%%%%%%%%%%%%%%%%%%%%% FRAME 4 %%%%%%%%%%%%%%%%%%%%%%%%%%%%%%%%%%%%%%%%%%%%%%%%
\begin{frame}{CERN - Oct 2018 (1)}

	\subfigs{\subfig{.5}[r]{PH5}}{\subfig{.5}[r]{F5}}
	
	\begin{itemize}\itemfill
		\item calibration in situ, operation with three planes
		\item weird low side of the distribution
		\item[] \textcolor{white}{\ra/}
	\end{itemize}
	
\end{frame}
%%%%%%%%%%%%%%%%%%%%%%%%%%%%%%%%%%%%%% FRAME 5 %%%%%%%%%%%%%%%%%%%%%%%%%%%%%%%%%%%%%%%%%%%%%%%%
\begin{frame}{CERN - Oct 2018 (2)}

	\subfigs{\subfig{.5}[r]{PH6}}{\subfig{.5}[r]{F6}}
	
	\begin{itemize}\itemfill
		\item re-calibration in situ, operation with three planes
		\item gives very similar result
		\item all calibrations in situ very similar.
	\end{itemize}
	
\end{frame}
%%%%%%%%%%%%%%%%%%%%%%%%%%%%%%%%%%%%%% FRAME 6 %%%%%%%%%%%%%%%%%%%%%%%%%%%%%%%%%%%%%%%%%%%%%%%%
\begin{frame}{CERN - Oct 2018 (3)}

	\subfigs{\subfig{.5}[r]{PH3}}{\subfig{.5}[r]{F3}}
	
	\begin{itemize}\itemfill
		\item calibration in situ, operation with single plane
		\item distribution looks very well, but still small negative contribution
		\item \ra small/negative signals can be related to small/degraded analogue signals...
	\end{itemize}
	
\end{frame}
%%%%%%%%%%%%%%%%%%%%%%%%%%%%%%%%%%%%%% FRAME 6 %%%%%%%%%%%%%%%%%%%%%%%%%%%%%%%%%%%%%%%%%%%%%%%%
\begin{frame}{CERN - Oct 2018 (4)}

	\subfigs{\subfig{.5}[r]{PH3}[Area with 3D columns.]}{\subfig{.5}[r]{PH3-1}[Area without 3D columns.]}
	
	\begin{itemize}\itemfill
		\item comparison between regions with and without 3D columns
	\end{itemize}
	
\end{frame}
