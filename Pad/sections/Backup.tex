\section{Backup}
%%%%%%%%%%%%%%%%%%%%%%%%%%%%%%%%%%%%%% FRAME 1 %%%%%%%%%%%%%%%%%%%%%%%%%%%%%%%%%%%%%%%%%%%%%%%%
\subsection{Timing Correction}
\begin{frame}[noframenumbering]{Ringbuffer}

	\subfigs{\subfig{.5}{ringbuffer}[Ringbuffer]}{\subfig{.5}[r]{TCAL}[Length of memory cells.]}
	
	\begin{itemize}\itemfill
		\item analogue signals of the diamonds constantly digitised and saved in ringbuffer
		\item overwrite old data once again at first cell
		\item once triggered data is saved starting from the current cell \ra trigger cell
		\item measure the length the of memory cells of the DRS4 (before every beam test)
		\item record trigger cell for every event
	\end{itemize}

\end{frame}
%%%%%%%%%%%%%%%%%%%%%%%%%%%%%%%%%%%%%% FRAME 2 %%%%%%%%%%%%%%%%%%%%%%%%%%%%%%%%%%%%%%%%%%%%%%%%
\begin{frame}[noframenumbering]{Peak Position}

	\subfigs{\subfig{.5}[r]{PeakPos}[no correction]}{\subfig{.5}[r]{PeakPos10}[with correction]}
	
	\begin{itemize}\itemfill
		\item timing of the signals should be fixed and determined by the scintillator
		\item non-corrected peak time distribution resembles cell size distribution
		\item correcting for the different cell sizes \ra strong improvement in timing
	\end{itemize}

\end{frame}
%%%%%%%%%%%%%%%%%%%%%%%%%%%%%%%%%%%%%% FRAME 3 %%%%%%%%%%%%%%%%%%%%%%%%%%%%%%%%%%%%%%%%%%%%%%%%
\begin{frame}[noframenumbering]{Fine Correction}

	\subfigs{\subfig{.5}[r]{PeakTC}[Dependence on trigger cell.]}{\subfig{.5}[r]{FineCorr}[fine correction]}
	
	\begin{itemize}\itemfill
		\item after drs4 time correction \ra still timing depends periodically on the trigger cell (why?)
		\item fit with periodic function with known period 
	\end{itemize}

\end{frame}
%%%%%%%%%%%%%%%%%%%%%%%%%%%%%%%%%%%%%% FRAME 4 %%%%%%%%%%%%%%%%%%%%%%%%%%%%%%%%%%%%%%%%%%%%%%%%
\begin{frame}[noframenumbering]{Timing Correction + Cut}

	\subfigs{\subfig{.5}[r]{TCor}[All corrections]}{\subfig{.5}[r]{TCorFit}[Timing cut]}
	
	\begin{itemize}\itemfill
		\item achieve \SI{\sim500}{\pico\second} timing resolution
		\item exclude signals outside \SI{3}{\sigma}) of this distribution
		\begin{itemize}
			\item wrong timing means something went wrong in the data-taking or the waveform is bad
		\end{itemize}
	\end{itemize}

\end{frame}
%%%%%%%%%%%%%%%%%%%%%%%%%%%%%%%%%%%%%% FRAME 0 %%%%%%%%%%%%%%%%%%%%%%%%%%%%%%%%%%%%%%%%%%%%%%%%
\subsection{Bucket Cut}
\begin{frame}[noframenumbering]{Origin}

	\only<1>{\fig{.5}{Bucket}}
	\only<2>{\fig{.5}{bucketwf}}
	
	\begin{itemize}\itemfill
		\item bunch spacing of PSI (\SI{19.7}{\nano\second}) small than clock cycle of fast-OR (\SI{25}{\nano\second})
		\item scintillator area \SI{\sim10}{times} larger than active trigger area
		\item within one clock cycle of \SI{25}{\nano\second}:
		\begin{itemize}
			\item \bad{one particle only hits the scintillator}
			\item \good{second particle hits the telescope and the diamond}
		\end{itemize}
		\item \ra no signal in signal region!
	\end{itemize}
	
\end{frame}
%%%%%%%%%%%%%%%%%%%%%%%%%%%%%%%%%%%%%% FRAME 1 %%%%%%%%%%%%%%%%%%%%%%%%%%%%%%%%%%%%%%%%%%%%%%%%
\begin{frame}[noframenumbering]{Bucket Pedestal}

	\fig[r]{.5}{Bucket1}
	
	\begin{itemize}\itemfill
		\item flat lines only when the highest peak is in the bunch after the trigger
	\end{itemize}
		
\end{frame}
%%%%%%%%%%%%%%%%%%%%%%%%%%%%%%%%%%%%%% FRAME 2 %%%%%%%%%%%%%%%%%%%%%%%%%%%%%%%%%%%%%%%%%%%%%%%%
\begin{frame}[noframenumbering]{Bucket Cut}

	\vspace*{-15pt}\fig[r]{.7}{Bucket2}
	
	\begin{itemize}\itemfill
		\item fit signal distribution when signal in the bunch after the trigger is higher
		\item signal and background well separated
		\item shift threshold and minimise the error on the signal
	\end{itemize}
		
\end{frame}%%%%%%%%%%%%%%%%%%%%%%%%%%%%%%%%%%%%%% FRAME 0 %%%%%%%%%%%%%%%%%%%%%%%%%%%%%%%%%%%%%%%%%%%%%%%%
\subsection{Event Selection}
\begin{frame}[noframenumbering]{Saturated}

	\fig[r]{.4}{sat}

	\begin{itemize}\itemfill
		\item DRS4 signal range: [-500, +500]\, mV
		\item exclude saturated waveforms \ra full pulse height information lost
		\item main source should be protons
		\item 17/200000 events in example above
	\end{itemize}
	
\end{frame}
%%%%%%%%%%%%%%%%%%%%%%%%%%%%%%%%%%%%%% FRAME 1 %%%%%%%%%%%%%%%%%%%%%%%%%%%%%%%%%%%%%%%%%%%%%%%%
\begin{frame}[noframenumbering]{Pulser}

	\only<1>{\fig[r]{.4}{PW}}
	\only<2>{\fig[r]{.4}{pul}}

	\begin{itemize}\itemfill
		\item<1-> use pulser as a reference signal
		\item<1-> tag pulser events by extra channel of the DRS4
		\item<2-> exclude these event since they don't have a diamond signal
		\item<2-> use for pulser analysis to compare to diamond signal
	\end{itemize}
	
\end{frame}
%%%%%%%%%%%%%%%%%%%%%%%%%%%%%%%%%%%%%% FRAME 2 %%%%%%%%%%%%%%%%%%%%%%%%%%%%%%%%%%%%%%%%%%%%%%%%
\begin{frame}[noframenumbering]{Event Range}
	
	\vspace*{-10pt}\fig[r]{.5}{ER}

	\begin{itemize}\itemfill
		\item until October 2015 \ra beam shutter opened after run was started
		\begin{itemize}
			\item unstable conditions
			\item exclude first five minutes of the run\vspace*{5pt}
		\end{itemize}
		\item past October 2015 exclude first minute as safety margin
		\begin{itemize}
			\item sometimes small adjustments made (e.g. collimator changed too late)
		\end{itemize}

	\end{itemize}
	
\end{frame}
%%%%%%%%%%%%%%%%%%%%%%%%%%%%%%%%%%%%%% FRAME 3 %%%%%%%%%%%%%%%%%%%%%%%%%%%%%%%%%%%%%%%%%%%%%%%%
\begin{frame}[noframenumbering]{Beam Interruption}
	
	\only<1>{\fig[r]{.4}{FP}}
	\only<2>{\fig[r]{.4}{FP1}}

	\begin{itemize}\itemfill
		\item<1-> usually short beam interruption every \SI{5}{\minute} at PSI + other interruption
		\item<2-> particle rate slowly ramps up after interruption
		\item<2-> exclude events when rate drops less than \SI{40}{\%} + \SI{5}{\second} before
		\item<2->  until rate is larger than \SI{40}{\%} + \SI{20}{\second} after this
		\item<2-> let pulse height adjust after beam interruption (safety margin)
	\end{itemize}
	
\end{frame}
%%%%%%%%%%%%%%%%%%%%%%%%%%%%%%%%%%%%%% FRAME 4 %%%%%%%%%%%%%%%%%%%%%%%%%%%%%%%%%%%%%%%%%%%%%%%%
\begin{frame}[noframenumbering]{Tracks}
	
	\fig[r]{.6}{OCs}

	\begin{itemize}\itemfill
		\item only use events with exactly one track
		\item require one and only one cluster per plane
	\end{itemize}
	
\end{frame}
%%%%%%%%%%%%%%%%%%%%%%%%%%%%%%%%%%%%%% FRAME 5 %%%%%%%%%%%%%%%%%%%%%%%%%%%%%%%%%%%%%%%%%%%%%%%%
\begin{frame}[noframenumbering]{Pedestal Sigma}
	
	\subfigs{\subfig{.5}[r]{PD}}{\subfig{.5}[r]{PDL}}

	\begin{itemize}\itemfill
		\item exclude pedestals outside the 3 sigma region
		\item baseline shifts
		\item bad waveforms
	\end{itemize}
	
\end{frame}
%%%%%%%%%%%%%%%%%%%%%%%%%%%%%%%%%%%%%% FRAME 9 %%%%%%%%%%%%%%%%%%%%%%%%%%%%%%%%%%%%%%%%%%%%%%%%
\begin{frame}[noframenumbering]{$\upchi^2$}

	\subfigs{\subfig{.5}[r]{ChiX}}{\subfig{.5}[r]{ChiY}}
	
	\begin{itemize}\itemfill
		\item exclude the bad tracks
	\end{itemize}
		
\end{frame}
%%%%%%%%%%%%%%%%%%%%%%%%%%%%%%%%%%%%%% FRAME 12 %%%%%%%%%%%%%%%%%%%%%%%%%%%%%%%%%%%%%%%%%%%%%%%
\begin{frame}[noframenumbering]{Tracking Angle}

	\only<1>{\subfigs{\subfig{.5}[r]{TAX1}}{\subfig{.5}[r]{TAY1}}}
	\only<2>{\subfigs{\subfig{.5}[r]{TAX}}{\subfig{.5}[r]{TAY}}}
	
	\begin{itemize}\itemfill
		\item only accept tracks with small angles
		\item<2-> angle only very slightly changes with rate
	\end{itemize}
		
\end{frame}
%%%%%%%%%%%%%%%%%%%%%%%%%%%%%%%%%%%%%% FRAME 13 %%%%%%%%%%%%%%%%%%%%%%%%%%%%%%%%%%%%%%%%%%%%%%%
\begin{frame}[noframenumbering]{Fiducial Cut}

	\subfigs{\subfig{.5}[r]{fid}}{\subfig{.5}[r]{fid1}}
	
	\begin{itemize}\itemfill
		\item select area of the diamond
		\item find first and last bin when signal drops lower than \SI{93}{\%} of the maximum value
		\item interpolate with the adjacent bins when threshold is exactly hit
		\item adjust manually if it fails or still pedestal left
	\end{itemize}
		
\end{frame}
